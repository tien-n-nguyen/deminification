%{\color{blue}{
\subsection{{\bf Comparison on Name Prediction in JS (RQ2)}}
\label{empirical-rq-js}

\begin{figure}[t] %[thbp]
\begin{center}
\includegraphics[width=3.2in]{figures/name-JS-prediction-result-2} %name-JS-prediction-result
\vspace{-8pt}
\caption{Top-1 Accuracy on Name Prediction in JS (RQ2)}
\label{name-JS-prediction-result}
\end{center}
\end{figure}

As seen in Figure~\ref{name-JS-prediction-result}, for all variables in
minified JS code, {\tool} achieves high top-1 accuracy of {\bf 81.6\%}:
i.e., in 4 out 5 cases, it can recover the correct variable names
with a single prediction. The relative improvements in top-1 accuracy
for all variables over JSNice, JSNaughty, and JSNeat are
%{\bf 46.6\%, 34.0\%}, and {\bf 5.4\%},
{\bf 49.7\%, 36.9\%,} and {\bf 7.7\%},
respectively.  The absolute improvements in top-1 accuracy over those
state-of-the-art approaches are from 5.8\%--27.1\%.
%4.1\%--25.4\%.

Considering only local variables, in
%{\bf 74.8\%}
{\bf 76.8\%} of them, {\tool} correctly predicts their original names
with a single result. The relative improvements in top-1 accuracy in
recovering local variables' names over JSNice, JSNaughty, and JSNeat
are
%{\bf 77.4\%, 52.7\%}, and {\bf 6.8\%},
{\bf 84.6\%, 59.0\%}, and {\bf 11.1\%}, respectively.  As seen in
Table~\ref{name-JS-result}, the comparison result is also consistent
for top-1, top-3, and top-5 accuracies.
%The absolute improvements in top-1 accuracy over those
%state-of-the-art approaches are from 7.7\%--35.2\%.


\begin{table}[t]%[thbp]
  \caption{Comparison on Name Prediction in JS (RQ2)}
  \vspace{-8pt}
	\begin{center}
		\small
		\renewcommand{\arraystretch}{1} \begin{tabular}{|p{1.9cm}<{\centering}|p{0.65cm}<{\centering}|p{0.65cm}<{\centering}|p{0.65cm}<{\centering}|p{0.65cm}<{\centering}|p{0.65cm}<{\centering}|p{0.65cm}<{\centering}|}
			
			\hline
                       & \multicolumn{2}{c|}{Top-1}         & \multicolumn{2}{c|}{Top-3}         & \multicolumn{2}{c|}{Top-5} \\
			\hline
                       & Local & All & Local & All & Local & All  \\ 
			\hline
		        JSNice~\cite{JSNice2015} &  41.6    & 54.5  & 56.4 &    68.2   & 64.2      &   72.4    \\
			JSNaughty~\cite{JSNaughty2017}  &   48.3   &  59.6    &  64.1    &  74.5    &  71.8    &   79.6    \\
			JSNeat~\cite{icse19}  &   69.1   &  75.8    &  69.6    & 79.5     &  76.9    & 86.0     \\
			\hline
			{\bf {\tool}} & 76.8 & 81.6 & 76.2 & 85.6 & 83.3 & 91.8 \\
			\hline
		\end{tabular}
		\label{name-JS-result}
	\end{center}
\end{table}

As seen in Figures~\ref{name-prediction-result}
and~\ref{name-JS-prediction-result}, {\tool} improves over the
baselines in both languages. However, its relative improvement over
the top baseline, JSNeat~\cite{icse19}, for Python is higher than that
for JS. The reason is that JS is a weakly typed language. In JS
code, the types of variables do not need to be specified, and can be
changed (that is, type information is not always available). {\tool}
is less effective in those cases, while JSNeat~\cite{icse19}, an IR
approach, is still effective since the variable names might be seen in
the JS dataset.

%to specify what type of information will be stored in a variable in
%advance.

%The reason is that JSNeat, an IR approach, was able to find more
%variable names in the JS dataset than in the Python dataset.  Note
%that JSNeat was built on JS code. Moreover, there are less variable
%names that need predictions in JS dataset than in Python dataset.

%}}
