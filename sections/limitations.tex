\vspace{-2pt}
\subsubsection{Limitations}
\label{sec:limitations}

%{\color{blue}{

First, {\tool} does not predict names well for the original code with
the short and meaningless names, e.g., \code{i}, \code{j}, etc. In
this case, the variable type could not help because the name and type
are not in accordance. The relations among the variables in the
context do not help either because there does not exist the
natural-language semantic connections among their names. The following
example shows such a case: the variable \code{r} at line 2 was
minified into \code{e} at line 6.  {\tool} failed this case and JSNeat
predicted correctly as it has seen the code with the same names. %});
\begin{lstlisting}[basicstyle=\footnotesize\sffamily, stepnumber=1, numbers=left, numbersep=-10pt, framexleftmargin=-2mm,
    framexrightmargin=-2mm]
    if (word.indexOf('G') === 0 || word.indexOf('M') === 0) {
      r = _find(SMOOTHIE_MODAL_GROUPS, (group) => {
        return _includes(group.modes, word); ...
    // Minified code
    if(s.indexOf("G")===0||s.indexOf("M")===0){
      e =_find(SMOOTHIE_MODAL_GROUPS,t=>{
        return _includes(t.modes,s); ...
\end{lstlisting}
Second, {\tool} does not produce well long variable names with several
sub-tokens.The next example shows a case in which the variable
\code{resetDocumentPropertyIsSet} (line 2) was minified
into \code{t} (line 6). {\tool} predicted an incorrect
name \code{resetDocumentOwnProperty} with its current decoder. JSNeat
has seen the same name in the corpus, thus, predicted it correctly.
For {\tool}, a solution could be to replace the GRU decoder with a
better model that can predict the length of the name and the name
itself.
\begin{lstlisting}[basicstyle=\footnotesize\sffamily, stepnumber=1, numbers=left, numbersep=-10pt, framexleftmargin=-2mm,
    framexrightmargin=-2mm]
    function shouldResetDoc(config) {
      var resetDocumentPropertyIsSet = config.hasOwnProperty("...");
      return config.resetDocument || !resetDocumentPropertyIsSet; ...
    // Minified code
    function shouldResetDoc(e) {
      var t = e.hasOwnProperty("...");
      return e.resetDocument || !t; ...
\end{lstlisting}
Third, for the type prediction, {\tool} works not so well for
the user-defined types with long names. Fourth, in some cases,
generic types (e.g., String) did not help learn variable names.
%the types did not help learn the variable name because the types are
%generic and any variable name could be used.

Finally, we currently implemented {\tool} for two languages: Python
and JS. However, the approach is general for any language with a
weak/strong type system in which it works better for a strong one. To
expand for a new such language, one just needs to have a parser and a
representation graph building module for that language. The graphs
will be fed to the same architecture model.
%}}

%{\color{blue}{Yi: we could find two examples: 1) original names are
%meaningless, e.g., i, j. etc; 2) the long names with multiple
%subtokens.}}
