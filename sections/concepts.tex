\section{Important Concepts}
\label{concepts:sec}

\begin{definition}{\bf [Attributes and Behaviors]}
The properties of a variable are not minified: its accessible fields
representing the attributes and its accessible methods representing
the behaviors performed by the corresponding object.
\end{definition}

From Observation 1, we can rely on the names of the fields and those
of the methods accessed from a variable as an important pivot in order
to recover the name of the variable. A model can encounter the same
names for the fields and methods, it can learn to derive the name of
the variable under study because the names of the variable and those
of the methods and fields in the original code are in harmony with one
another. For example, in Figure~\ref{example_sim}, at line 7 and line
9, we will explore the variable \code{r} in its field accesses and
method calls \code{r.cloneRange()}, \code{r.startOffset}, and
\code{r.endOffset}, because the names for those fields and methods are
intact after code minification. We denote an instance of field access
and method call as a triple $(v, p, t)$, where $v$ is the variable,
$p$ is the name of the field or method, and $t$ is either
\code{fieldAccess} or \code{methodCall}, e.g., $(r, cloneRange,
methodCall)$ and $(r, startOffset, fieldAccess)$.


%Tien

\begin{definition}{\bf [Argument Relation]}
  A variable $v$ is said to have an argument relation with a method
  $m$ if it is used as an argument of a call to that method as in
  $o.m(...,v,...)$.
\end{definition}

\begin{definition}{\bf [Assignment Relation]}
  A variable $v$ is said to have an assignment relation with a method
  $m$ or a field $f$ if it is used as a left-hand side in an
  assignment from a method call or a field access as in $v =
  o.m(...)$, or $v = o.f$.
\end{definition}


%
%We focus on the roles of a variable used {\em as an argument in a method call}
% or {\em receiving the value returned by a method call or field access}.
%
%If we have $o.m(...,v,...)$, $v = o.m(...)$, or $v = o.f$,
%where $m$ is a method call and $f$ is a field access,
%then there exist
%the role relations between $v$ and $m$, and between $v$ and $f$.

The idea is that the name of the minified variable $v$ in the original
code is often in accordance with the names of the method or the field
in such an assignment or an argument. For example, \code{range} and
\code{getRangeAt} are in accordance with each other in \code{range =
  selection.getRangeAt(0)}; or \code{selectNodeContents} and
\code{root} are in accordance with each other in
\code{preSelectionRange.selectNodeContents(root)}.  We will use the
triple notations $(v, p, argument)$, $(v, p, assignment)$, and
$(v, f, assignment)$ to denote those three cases, where $v$ is a
variable, $p$ is a method name, and $f$ is a field name.

%The rationale is that the names of $m$ and its argument are often in
%conformance with each other, \eg
%\texttt{getData(contentType)}. Similar rationale is applied to the
%above assignments to $v$.

%A variable could be received a value of a field/method, or could be a
%input of an method invocation. For instance, in
%figure \ref{example_sim}, variable \texttt{r} receives the result of
%an assignment to the boolean expression $t.clipboardData ||
%a.clipboardData || e.dataTransfer$. In case of variable \texttt{f}, it
%is an argument of function \texttt{getData()} and the value
%of \texttt{getData()} is influenced by \texttt{f}.

%A role relation between a variable \texttt{v} and a field/method
%\texttt{p} is denoted by a triple $(v, p, t)$, where
%\texttt{t} is the type of role relation. A role relation could
%be either \texttt{argument} or \texttt{assignment}.

%We expect this role relation to contribute significantly to the name
%recovery process.
%%
%%This type of relation also contributes remarkably in reducing the name
%%searching space when recovering a variable.
%For example, a variable \texttt{x} in a minified code is assigned with
%the value of a field access to \texttt{httpBody} and is used as an
%argument in a call to \texttt{setBodyHttp()}. In our dataset, there
%are only 19 variable names that are assigned with the value
%of \texttt{httpBody}, while the number
%%of variable names is used for method call
%with regard to \texttt{setBodyHttp()} is 11.
%%The figure for names having both two relations is only 5.
%Then, there are only 5 candidate names that satisfy both conditions.
%Thus, our algorithm could reduce the number of candidates from +200K
%to only 5.
%%Therefore, the number of variable names could be recovered
%%for \texttt{x} reduced down to 5 in comparison with about 200k
%%variable names of our corpus.



%In our scope, we consider 2 common data dependency between
%variable \texttt{x} with fields, methods, which is shown in
%table \ref{table:DataDep} as below.

%\begin{table}[h!]
%\begin{tabular}{|c|c|c|c| }
% \hline Sample & Variable & Methods/Fields & Dependency Type \\
% [0.5ex] \hline x = Http.httpBody & x & httpBody &
% Assignment \\ \hline t.setBodyHttp(x) & x & setBodyHttp() &
% Argument \\ \hline
%\end{tabular}
%\caption{Data dependency}
%\label{table:DataDep}
%\end{table}


\begin{definition}{\bf [Relation Graph]}~\cite{icse19}
A relation graph (RG) is a directed graph
%in the shape of a star to represent the single-variable usage context
%of $v$ with regard to its property and role relations with 
%the fields and methods in its usage.
%
in which each node of the RG represents a variable. The connected
nodes represent the methods/fields in method calls or field accesses,
respectively, and are labeled with their names. Edges represent
relations among nodes and are labeled with relation types.
\end{definition}

%Figure~\ref{SG_sample_ano} shows the relation graph of the variable
%\texttt{r}, which includes a set of property relations:
%$( \texttt{r}, \texttt{types}, \texttt{fieldAccess})$, $( \texttt{r},
%\texttt{getData}, \texttt{fieldAccess})$, $( \texttt{r},
%\texttt{getData()}, \texttt{methodCall})$,
%
%and a set of role relations: $( \texttt{r}, \texttt{clipboardData},
%\texttt{assignment})$, $( \texttt{r}, \texttt{dataTransfer},
%\texttt{assignment})$ in our example.

\begin{figure}[t]
	\begin{center}
		\includegraphics[width=0.9\columnwidth]{figures/relation-graph}
		\caption{The Relation Graph for Figure~\ref{example_sim}}
		\label{rel-graph}
	\end{center}
\end{figure}

Figure~\ref{rel-graph} shows the relation graph for the variables in
the code in Figure~\ref{example_sim}. For example, there are an
$assign$ edge from the variable \code{r} to the node
\code{getRangeAt}, and a $methodcall$ edge from \code{q} to
\code{getRangeAt} because we have \code{r = q.getRangeAt(0)} at line
7.




\begin{figure}[t]
  \small
  \begin{eqnarray*}
    \theta \in Type (\Theta) &::=& \gamma | \, \alpha [\theta, ..., \theta] \, | \, u \, | \, \mathbf{None} \, | \, \mathbf{type}\\
  \gamma \in \, Elementary \, Type (\Gamma) &::=& \mathbf{int} \, | \, \mathbf{float} \, | \, \mathbf{str} \, | \, \mathbf{bool} \, | \, \mathbf{bytes}\\
  \alpha \in Generic \, Type (A) &::=& \mathbf{List} \, | \, \mathbf{Tuple} \, | \, \mathbf{Dict} \, | \, \mathbf{Set} \, |\\
  & & \, \mathbf{Callable} \, | \, \mathbf{Generator} \, | \, \mathbf{Union}\\
  b \in \, Builtin \, Type (B) &::=& \gamma \, | \, \alpha[\theta]\\
  u \in \, User \, Defined \, Type (U) &::=& all \, classes \, and \, named \, tuples\\
  o \in \, Overloading \, User \, Def \, Type (O) &::=& all \, classes \, with \, operator \\
  & & overloading \, in \, code
  \end{eqnarray*}
  \vspace{-18pt}
\caption{Types in Python}
\label{python-types}
\end{figure}

Figure~\ref{python-types} shows the type system in
Python~\cite{type-graph-icse22}. Due to space limit, we do not include
the type system for JavaScript. To represent the dependencies among
the types, we adopt the type dependency graph (TDG) from
HiTyper~\cite{type-graph-icse22}.

\begin{definition}{\bf Type Dependency Graph]}~\cite{type-graph-icse22}
    \label{tdg-def}
A Type Dependency Graph is a graph $G$ = $(N,E)$ in which $N$ is the
set of nodes representing all the variables and expresions, and $E$ is
the set of edges from $n_i$ to $n_j$ indicating that the type of $n_j$
can be derived from the type of $n_i$ by the type inference rules in
the type system.
\end{definition}

For example, from line 10 of Figure~\ref{example_org}, \code{start} =
\code{preSelectionRange.} \code{toString().} \code{length}, our model
can build the dependency between the type of \code{start} from the
type of \code{length} in the \code{String} class (which is
\code{int}). As another example, there is a type dependency between
the return type of \code{getIndexRelativeToAdjacentEmptyBlocks} and
the type of the variable \code{emptyBlocksIndex} at line 23, and
another dependency between the type of \code{emptyBlocksIndex} and the
type of \code{selectState.emptyBlocksIndex} at line 25.  Connecting
all the dependencies among the types of variables and expressions, we
have the type dependency graph for the code as defined in
Definition~\ref{tdg-def}.

We leverage the graph building algorithm from HiTyper to parse the
code and transform every variable occurrence and expression into nodes
and maintains type dependencies between them. We also extend the
notion and the algorithm to build the type dependency graph (TDG) for
the JavaScript code. Details on TDG is in~\cite{type-graph-icse22}.
