Figure~\ref{example2} shows another example in the project
\code{EventBus} in our dataset. At line 12, our model can recognize
the variable \code{D} used in a \code{for} loop with an array of
\code{listeners}, thus, can assign for \code{D} the name
\code{listener}. From line 8, our model can learn that \code{B} is of
the type \code{Subcribable} and at line 13 and it knows that \code{B}
is related to \code{D} (\code{listener}). Thus, from {\em Subcribable}
and \emph{listener}, it could derive the name for \code{B} as
\code{event}. At line 9, if our model can learn that \code{type} is to
return the class for \code{B}, it can assign 
\code{event\_class} for \code{C} due to the assignment.

\begin{figure}[t]
	\centering \lstset{ numbers=left, numberstyle= \tiny,
	keywordstyle= \color{blue!70},
	commentstyle= \color{red!50!green!50!blue!50},
	frame=shadowbox, rulesepcolor= \color{red!20!green!20!blue!20}
	, xleftmargin=1.5em,xrightmargin=0em, aboveskip=1em,
	framexleftmargin=1.5em, numbersep= 5pt, language=Python,
	basicstyle=\scriptsize\ttfamily,
	numberstyle=\scriptsize\ttfamily, emphstyle=\bfseries,
	moredelim=**[is][\color{red}]{@}{@}, escapeinside= {(*@}{@*)}
	} \begin{lstlisting}[]
def post(self, (*@{\color{orange}{event: Subscribable}@*)) -> None:
  event_class = type(event)
  if event_class not in self._listeners:
    return
  (*@{\color{violet}{for listener in self.\_listeners[event\_class]:}@*)
    (*@{\color{orange}{listener(event)}@*)
// (*@{\color{green}{Minified code}@*)
def post(self, (*@{\color{orange}{B:Subscribable}@*))->None:
  C=type(B)
  if C not in self._listeners:
    return
  (*@{\color{violet}{for D in self.\_listeners[C]:}@*)
    (*@{\color{orange}{D(B)}@*)
\end{lstlisting}
\vspace{-16pt}
\caption{Another Correct Prediction Example by {\tool}}
\label{example2}
\end{figure}
