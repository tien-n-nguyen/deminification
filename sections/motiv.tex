\section{Motivating Example}
\label{example_section}



\begin{figure}[t]
	\centering
	\lstset{
		numbers=left,
		numberstyle= \tiny,
		keywordstyle= \color{blue!70},
		commentstyle= \color{red!50!green!50!blue!50},
		frame=shadowbox,
		rulesepcolor= \color{red!20!green!20!blue!20} ,
		xleftmargin=1.5em,xrightmargin=0em, aboveskip=1em,
		framexleftmargin=1.5em,
                numbersep= 5pt,
		language=Java,
    basicstyle=\scriptsize\ttfamily,
    numberstyle=\scriptsize\ttfamily,
    emphstyle=\bfseries,
                moredelim=**[is][\color{red}]{@}{@},
		escapeinside= {(*@}{@*)}
	}
	\begin{lstlisting}[]
function exportSelection(root, doc) {
  if (!root) {
      return null;
  }
  var selectionState = null, selection = doc.getSelection();
  if (selection.rangeCount > 0) {
      var range = selection.getRangeAt(0), preSelectionRange = range.cloneRange(), start;
      preSelectionRange.selectNodeContents(root);
      preSelectionRange.setEnd(range.startContainer, range.startOffset);
      start = preSelectionRange.toString().length;
      selectionState = {
          start: start,
          end: start + range.toString().length
      };
      if (this.doesRangeStartWithImages(range, doc)) {
          selectionState.startsWithImage = true;
      }
      var trailingImageCount = this.getTrailingImageCount(root, selectionState, range.endContainer, range.endOffset);
      if (trailingImageCount) {
        selectionState.trailingImageCount = trailingImageCount;
      }
      if (start !== 0) {
        var emptyBlocksIndex = this.getIndexRelativeToAdjacentEmptyBlocks(doc, root, range.startContainer, range.startOffset);
        if (emptyBlocksIndex !== -1) {
          selectionState.emptyBlocksIndex = emptyBlocksIndex;
        }...
        \end{lstlisting}
\vspace{-12pt}
\caption{An Original Code from a Project in GitHub}
\label{example_org}
\end{figure}


\begin{figure}[t]
	\centering
	\lstset{
		numbers=left,
		numberstyle= \tiny,
		keywordstyle= \color{blue!70},
		commentstyle= \color{red!50!green!50!blue!50},
		frame=shadowbox,
		rulesepcolor= \color{red!20!green!20!blue!20} ,
		xleftmargin=1.5em,xrightmargin=0em, aboveskip=1em,
		framexleftmargin=1.5em,
                numbersep= 5pt,
		language=Java,
    basicstyle=\scriptsize\ttfamily,
    numberstyle=\scriptsize\ttfamily,
    emphstyle=\bfseries,
                moredelim=**[is][\color{red}]{@}{@},
		escapeinside= {(*@}{@*)}
	}
	\begin{lstlisting}[]
function exportSelection(w, b) {
  if (!w) {
    return null;
  }
  var p = null, q = b.getSelection();
  if (q.rangeCount > 0) {
    var r = q.getRangeAt(0), d = r.cloneRange(), m;
    d.selectNodeContents(w);
    d.setEnd(r.startContainer, r.startOffset);
    m = d.toString().length;
    p = {
          start: m,
          end: m + r.toString().length
        };
    if (this.doesRangeStartWithImages(r, b)) {
       p.startsWithImage = true;
    }
    var a = this.getTrailingImageCount(w, p, r.endContainer, r.endOffset);
    if (a) {
      p.trailingImageCount = a;
    }
    if (m !== 0) {
      var y = this.getIndexRelativeToAdjacentEmptyBlocks(b, w, r.startContainer, r.startOffset);
      if (y !== -1) {
        p.emptyBlocksIndex = y;
      }...
\end{lstlisting}
\vspace{-12pt}
\caption{The Minified Code for the Code in Figure~\ref{example_org}}
\label{example_sim}
\end{figure}

Let us start with a real-world example to motivate our approach.
Figures~\ref{example_org} and~\ref{example_sim} show the original and
minified versions~of the JS function \code{exportSelection}.  The
function is aimed to export/retrieve the selection from a document.
%
In the minified code, all local variables were randomly renamed with
short and meaningless names, \eg \code{root} becomes \code{w},
\code{doc} becomes \code{b}, etc. by the minification tool,
\eg UglifyJS~\cite{uglifyJS}. This makes them difficult to
comprehend.

%\subsection{Observations}

We aim to recover the names of the variables in the minified
code. Such process is not trivial and affected by multiple factors.
Let us explain the following observations then to motivate our
solution:

\vspace{1pt} \textbf{Observation 1.} {\em The mutual impact between
  variable name learning and variable type learning.} If a model
learns correctly the type of a variable, it would help to learn better
the name of the variable and vice versa. Let us consider the name
\code{emptyBlocksIndex} for the variable \code{y} at line 24 in
Figure~\ref{example_sim}: \code{if (y != -1)}. From that comparison, a
model could learn that \code{y} is of the type \code{int}. Knowing
that it is an integer, a model could combine that with the knowledge
learned from line 23 (\code{var y =
  this.getIndexRelativeToAdjacentEmptyBlocks(...)}), and predict the
name for \code{y} could be {\em ``Index''} or similar. On
the other hand, correct learning of a variable name can also benefit
for learning of its type. At line 18 of Figure~\ref{example_sim}, if
the name \code{trailingImageCount} is recovered for the variable
\code{a}, then its type is likely to be \code{int} if the model could
make sense of the sub-tokens \code{image} and \code{count}.

\vspace{1pt} \noindent {\bf Key Idea 1.} [{\bf Dual-task Learning
    between Name Prediction and Type Prediction}] {\em We leverage the
  duality between the learning to predict the variable names and the
  learning to predict the variable types.} In the original code, the
name of a variable should be natural (unsurprising) with respect to
the type of that variable. We build a name prediction model and a type
prediction one, and we apply a dual-task learning mechanism between
the two models.

\textbf{O2}. {\em The name of a variable can be affected by the names
  of the surrounding variables in a function.} Intuitively, because
multiple variables are used together to achieve the common task in the
function, their names are often consistent with one another. For
example, at line 7, the choice of the name \code{range} in the
recovery process could be influenced by the choice of the variable
\code{selection} as the call to \code{getRangeAt} is made on that
variable as in the statement \code{range} =
\code{selection.getRangeAt(0)}. The choice of the name
\code{preSelectionRange} could be affected by the choice of the
variable \code{range} due to the statement \code{preSelectionRange} =
\code{range.cloneRange()}.

\textbf{O3}. {\em The actual variable name must also be in accordance
  with the names of the accessed fields and called methods}. For
example, in the original code in Figure~\ref{example_org}, the
variable name \code{range} makes sense in \code{range.startOffset} and
\code{range.endOffset} because a range could have a start offset and
an ending offset. The rationale is that in the orignal code, for easy
code comprehension, developers tend to use the naming conventions and
meaningful names with respect to the surrounding names in the code. In
other words, the predicted name of a variable and those of its
properties (fields and methods) are in accordance. As another example,
\code{preSelectionRange} and \code{setEnd} are in accordance with each
other in \code{preSelectionRange.setEnd} at line 9. In fact, there exist
approaches to explore the concordance between the method's name and
the names of its arguments to detect name-based bugs in a
program~\cite{deepbugs-oopsla18}.

\textbf{O4}.{\em The fields and methods of a variable are kept intact
  after minification.} If the names of the fields and methods were
minified, the corresponding field accesses and method calls would not
be valid anymore. For example, \code{cloneRange()} in
\code{r.cloneRange()} at line 7 and \code{startContainer} in
\code{r.startContainer} at line 9 in Figure~\ref{example_org} are
unchanged in Figure~\ref{example_sim}. Due to that, a model can rely
on the names of those properties of a variable/object to predict the
variable's name.















