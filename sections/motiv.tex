\section{Motivating Example}
\label{example_section}

\begin{figure}[t]
	\centering
	\lstset{
		numbers=left,
		numberstyle= \tiny,
		keywordstyle= \color{blue!70},
		commentstyle= \color{red!50!green!50!blue!50},
		frame=shadowbox,
		rulesepcolor= \color{red!20!green!20!blue!20} ,
		xleftmargin=1.5em,xrightmargin=0em, aboveskip=1em,
		framexleftmargin=1.5em,
                numbersep= 5pt,
		language=Python,
    basicstyle=\scriptsize\ttfamily,
    numberstyle=\scriptsize\ttfamily,
    emphstyle=\bfseries,
                moredelim=**[is][\color{red}]{@}{@},
		escapeinside= {(*@}{@*)}
	}
	\begin{lstlisting}[]
function exportSelection(self, root, doc) {
  if not root: 
      return null
  selection = doc.getSelection()
  if selection.rangeCount > 0:
      range_ = selection.getRangeAt(0)
      preSelectionRange = range_.cloneRange()
      preSelectionRange.selectNodeContents(root)
      preSelectionRange.setEnd(range_.startContainer, range_.startOffset)
      start = len(str(preSelectionRange))
      selectionState = {
          "start": start,
          "end": start + len(str(range_))
      }
      if self.doesRangeStartWithImages(range_, doc):
          selectionState.startsWithImage = true
      (*@{\color{orange}{trailingImageCount}@*) = self.getTrailingImageCount(root, selectionState, range_.endContainer, range_.endOffset)
      if trailingImageCount:
        selectionState.trailingImageCount = trailingImageCount
      if start != 0:
        (*@{\color{violet}{emptyBlocksIndex = self.getIndexRelativeToAdjacentEmptyBlocks}@*)(doc, root, range_.startContainer, range_.startOffset)
        (*@{\color{violet}{if emptyBlocksIndex != -1:}@*) 
          selectionState.emptyBlocksIndex = emptyBlocksIndex
        ...
        \end{lstlisting}
\vspace{-12pt}
\caption{An Original Code from a Project in GitHub}
\label{example_org}
\end{figure}


%%%\begin{figure}[t]
%%%	\centering
%%%	\lstset{
%%%		numbers=left,
%%%		numberstyle= \tiny,
%%%		keywordstyle= \color{blue!70},
%%%		commentstyle= \color{red!50!green!50!blue!50},
%%%		frame=shadowbox,
%%%		rulesepcolor= \color{red!20!green!20!blue!20} ,
%%%		xleftmargin=1.5em,xrightmargin=0em, aboveskip=1em,
%%%		framexleftmargin=1.5em,
%%%                numbersep= 5pt,
%%%		language=Java,
%%%    basicstyle=\scriptsize\ttfamily,
%%%    numberstyle=\scriptsize\ttfamily,
%%%    emphstyle=\bfseries,
%%%                moredelim=**[is][\color{red}]{@}{@},
%%%		escapeinside= {(*@}{@*)}
%%%	}
%%%	\begin{lstlisting}[]
%%%function exportSelection(root, doc) {
%%%  if (!root) {
%%%      return null;
%%%  }
%%%  var selectionState = null, selection = doc.getSelection();
%%%  if (selection.rangeCount > 0) {
%%%      var range = selection.getRangeAt(0), preSelectionRange = range.cloneRange(), start;
%%%      preSelectionRange.selectNodeContents(root);
%%%      preSelectionRange.setEnd(range.startContainer, range.startOffset);
%%%      start = preSelectionRange.toString().length;
%%%      selectionState = {
%%%          start: start,
%%%          end: start + range.toString().length
%%%      };
%%%      if (this.doesRangeStartWithImages(range, doc)) {
%%%          selectionState.startsWithImage = true;
%%%      }
%%%      var trailingImageCount = this.getTrailingImageCount(root, selectionState, range.endContainer, range.endOffset);
%%%      if (trailingImageCount) {
%%%        selectionState.trailingImageCount = trailingImageCount;
%%%      }
%%%      if (start !== 0) {
%%%        (*@{\color{violet}{var emptyBlocksIndex = this.getIndexRelativeToAdjacentEmptyBlocks}@*)(doc, root, range.startContainer, range.startOffset);
%%%        (*@{\color{violet}{if (emptyBlocksIndex !== -1)}@*) {
%%%          selectionState.emptyBlocksIndex = emptyBlocksIndex;
%%%        }...
%%%        \end{lstlisting}
%%%\vspace{-12pt}
%%%\caption{An Original Code from a Project in GitHub}
%%%\label{example_org}
%%%\end{figure}

\begin{figure}[t]
	\centering
	\lstset{
		numbers=left,
		numberstyle= \tiny,
		keywordstyle= \color{blue!70},
		commentstyle= \color{red!50!green!50!blue!50},
		frame=shadowbox,
		rulesepcolor= \color{red!20!green!20!blue!20} ,
		xleftmargin=1.5em,xrightmargin=0em, aboveskip=1em,
		framexleftmargin=1.5em,
                numbersep= 5pt,
		language=Java,
    basicstyle=\scriptsize\ttfamily,
    numberstyle=\scriptsize\ttfamily,
    emphstyle=\bfseries,
                moredelim=**[is][\color{red}]{@}{@},
		escapeinside= {(*@}{@*)}
	}
	\begin{lstlisting}[]
function exportSelection(self, w, b) {
  if not w:
     return null
  q = b.getSelection()
  if q.rangeCount > 0:
    r = q.getRangeAt(0)
    d = r.cloneRange()
    d.selectNodeContents(w)
    d.setEnd(r.startContainer, r.startOffset)
    m = len(str(d))
    p = {
          "start": m,
          "end": m + len(str(r))
        };
    if self.doesRangeStartWithImages(r, b):
       p.startsWithImage = true
    (*@{\color{orange}{a = self.getTrailingImageCount}@*)(w, p, r.endContainer, r.endOffset)
    if a:
      p.trailingImageCount = a
    if m !== 0:
      (*@{\color{violet}{y = self.getIndexRelativeToAdjacentEmptyBlocks}@*)(b, w, r.startContainer, r.startOffset)
      (*@{\color{violet}{if y != -1:}@*)
        p.emptyBlocksIndex = y
      ...
\end{lstlisting}
\vspace{-12pt}
\caption{The Minified Code for the Code in Figure~\ref{example_org}}
\label{example_sim}
\end{figure}


%%%\begin{figure}[t]
%%%	\centering
%%%	\lstset{
%%%		numbers=left,
%%%		numberstyle= \tiny,
%%%		keywordstyle= \color{blue!70},
%%%		commentstyle= \color{red!50!green!50!blue!50},
%%%		frame=shadowbox,
%%%		rulesepcolor= \color{red!20!green!20!blue!20} ,
%%%		xleftmargin=1.5em,xrightmargin=0em, aboveskip=1em,
%%%		framexleftmargin=1.5em,
%%%                numbersep= 5pt,
%%%		language=Java,
%%%    basicstyle=\scriptsize\ttfamily,
%%%    numberstyle=\scriptsize\ttfamily,
%%%    emphstyle=\bfseries,
%%%                moredelim=**[is][\color{red}]{@}{@},
%%%		escapeinside= {(*@}{@*)}
%%%	}
%%%	\begin{lstlisting}[]
%%%function exportSelection(w, b) {
%%%  if (!w) {
%%%    return null;
%%%  }
%%%  var p = null, q = b.getSelection();
%%%  if (q.rangeCount > 0) {
%%%    var r = q.getRangeAt(0), d = r.cloneRange(), m;
%%%    d.selectNodeContents(w);
%%%    d.setEnd(r.startContainer, r.startOffset);
%%%    m = d.toString().length;
%%%    p = {
%%%          start: m,
%%%          end: m + r.toString().length
%%%        };
%%%    if (this.doesRangeStartWithImages(r, b)) {
%%%       p.startsWithImage = true;
%%%    }
%%%    var (*@{\color{orange}{a = this.getTrailingImageCount}@*)(w, p, r.endContainer, r.endOffset);
%%%    if (a) {
%%%      p.trailingImageCount = a;
%%%    }
%%%    if (m !== 0) {
%%%      (*@{\color{violet}{var y = this.getIndexRelativeToAdjacentEmptyBlocks}@*)(b, w, r.startContainer, r.startOffset);
%%%      (*@{\color{violet}{if (y !== -1)}@*) {
%%%        p.emptyBlocksIndex = y;
%%%      }...
%%%\end{lstlisting}
%%%\vspace{-12pt}
%%%\caption{The Minified Code for the Code in Figure~\ref{example_org}}
%%%\label{example_sim}
%%%\end{figure}

Let us start with a real-world example to motivate our approach.
Figures~\ref{example_org} and~\ref{example_sim} show the original and
minified versions of the function \code{exportSelection} in Python.
The function is aimed to export/retrieve the selection from a
document.
%
In the minified code, all local variables were randomly renamed by a
minification tool with short and meaningless names, \eg \code{root}
becomes \code{w}, \code{doc} becomes \code{b}, etc.
%\eg UglifyJS~\cite{uglifyJS}.
This makes the code difficult to comprehend.

%\subsection{Observations}

We aim to recover the names of the variables in the minified
code. Such process is not trivial and affected by multiple factors.
Let us explain the following observations then to motivate our
solution:

\vspace{2pt} \textbf{Observation 1.} {\em The mutual impact between
  variable name learning and variable type learning.} If a model
learns correctly the type of a variable, it would help to learn better
the name of the variable and vice versa. Let us consider the name
\code{emptyBlocksIndex} for the variable \code{y} at line 22 in
Figure~\ref{example_sim}: \code{if y != -1:}. From that comparison, a
model could learn that \code{y} is of the type \code{int}. Knowing
that it is an integer, a model could combine that with the knowledge
learned from line 21 (\code{y =
  self.getIndexRelativeToAdjacentEmptyBlocks(...)}), and predict the
name for \code{y} could be {\em ``Index''} or similar. On the other
hand, correct learning of a variable name can also benefit for
learning of its type. At line 17 of
Figures~\ref{example_org}--\ref{example_sim}, if the name
\code{trailingImageCount} is recovered for the variable \code{a},
its type is likely to be \code{int} if the model could make sense of
the sub-token \code{count} in that name \code{trailingImageCount}.

\vspace{2pt} \noindent {\bf Key Idea 1.} [{\bf Dual-task Learning
    between Name Prediction and Type Prediction}] {\em While focusing
  on recovering variable names in minified code, we leverage the
  duality between the learning to predict the variable names and the
  learning to predict the variable types.} In the original code, the
name of a variable should be natural (unsurprising) with respect to
the type of that variable. We build a name prediction model and a type
prediction one, and we apply a dual-task learning mechanism between
the two models.

\vspace{2pt}
\textbf{Observation 2}. {\em The name and type of a variable are
  affected by the names and types of the surrounding variables in a
  function.}  Intuitively, because multiple variables are used
together to achieve the task in the function, their names are
often consistent with one another. For example, at line 6, the choice
of the name \code{range\_} in the recovery process could be derived
%influenced
by the choice of the variable \code{selection} and the call to
\code{getRangeAt} as it is made on that variable as in the statement
\code{range\_} = \code{selection.getRangeAt(0)}. The choice of the name
\code{preSelectionRange} could be affected by the choice of the
variable \code{range\_} due to the statement \code{preSelectionRange} =
\code{range\_.cloneRange()}. Moreover, the type system in a programming
language always requires the concordance between the types of
variables. For example, the types of two sides of an assignment must
be consistent.

\vspace{2pt}
\noindent {\bf Key Idea 2.} [{\bf {\em ``Tell Me Your Friends, I'll Tell
    You Who You Are''}}]
%[{\bf Relations among Variables in a Function}]
          {\em A variable name or type are influenced by the
  names or the properties of the other variables having the relations
  with that variable} in the surrounding context. We {\bf treat the
  problem of variable name generation as predicting the missing
  features} in a graph neural network by leveraging Graph Convolution
Network - Missing Features (GCNmf)~\cite{GCNmf}.  We also use
Edge-Enhanced Graph Convolution Networks (EE-GCN)~\cite{ee-gcn} to
model different kinds of relations among the
variables and types in the function/method.


%For example, a model can be trained from the assignment
%\code{preSelectionRange} = \code{range.cloneRange()} to predict the
%variable name on the left-hand side of \code{X =
%range.cloneRange()}. In other cases, a model can learn the names that
%often appear together in several methods in the training corpus. For
%example, the variable names \code{selection}, \code{selectionState},
%\code{range}, and \code{preSelectionRange}.

\vspace{2pt} \textbf{Observation 3}. {\em The actual variable name
  must also be in accordance with the names of the accessed fields and
  called methods}. For example, in the original code in
Figure~\ref{example_org}, the variable name \code{range\_} makes sense
in \code{range\_.startOffset} and \code{range\_.endOffset} because a
range could have a starting offset and an ending offset. The rationale
is that in the orignal code, for easy comprehension, developers tend
to use the naming conventions and meaningful names with respect to the
surrounding names in the code. In other words, the predicted name of a
variable and those of its properties (fields and methods) are in
accordance. As another example, \code{preSelectionRange} and
\code{setEnd} are in accordance with each other in
\code{preSelectionRange.setEnd} ({\em ``Setting the end of the
  selected range''}). In fact, Pradel {\em et
  al.}~\cite{deepbugs-oopsla18} explore the concordance between the
method's name and the names of its arguments to detect name-based bugs
in a program.

% at line 9

\vspace{2pt}
\textbf{Observation 4}. {\em The fields and methods of a variable are
  kept intact after minification.} If the names of the fields and
methods were minified, the corresponding field accesses and method
calls would not be valid anymore. For example, \code{cloneRange()} in
\code{range\_.cloneRange()} at line 7 and \code{startContainer} in
\code{range\_.startContainer} at line 9 in Figure~\ref{example_org} are
unchanged in Figure~\ref{example_sim}. Thus, a model can rely
on the names of those properties of a variable to predict the
variable's name.

\vspace{2pt}
\noindent {\bf Key Idea 3.} [{\bf Properties of a Variable}] {\em The
  name of a variable~is in accordance with its own properties}
including the names of its fields in field accesses and the names of
its methods in the method calls on that variable. Moreover, the names
of fields and methods are~kept intact after minification, thus, a
model can rely on those names to predict the variables' names. For
example, a model can learn from the variables that have the field
accesses to \code{start\-Container}, \code{end\-Container},
\code{start\-Offset}, and \code{end\-Offset}, and have the method
calls to \code{clone\-Range()}.  It then uses that knowledge to
predict the name \code{range\_}.

%variable type and the name \code{range\_}.

In brief, to recover the name of a variable, a model could examine {\bf its
own properties}, {\bf its relations} with other variables and their
properties, and the relations among {\bf their types}.















