\section{Motivating Example}
\label{example_section}



\begin{figure}[t]
	\centering
	\lstset{
		numbers=left,
		numberstyle= \tiny,
		keywordstyle= \color{blue!70},
		commentstyle= \color{red!50!green!50!blue!50},
		frame=shadowbox,
		rulesepcolor= \color{red!20!green!20!blue!20} ,
		xleftmargin=1.5em,xrightmargin=0em, aboveskip=1em,
		framexleftmargin=1.5em,
                numbersep= 5pt,
		language=Java,
    basicstyle=\scriptsize\ttfamily,
    numberstyle=\scriptsize\ttfamily,
    emphstyle=\bfseries,
                moredelim=**[is][\color{red}]{@}{@},
		escapeinside= {(*@}{@*)}
	}
	\begin{lstlisting}[]
function exportSelection(root, doc) {
  if (!root) {
      return null;
  }
  var selectionState = null, selection = doc.getSelection();
  if (selection.rangeCount > 0) {
      var range = selection.getRangeAt(0), preSelectionRange = range.cloneRange(), start;
      preSelectionRange.selectNodeContents(root);
      preSelectionRange.setEnd(range.startContainer, range.startOffset);
      start = preSelectionRange.toString().length;
      selectionState = {
          start: start,
          end: start + range.toString().length
      };
      if (this.doesRangeStartWithImages(range, doc)) {
          selectionState.startsWithImage = true;
      }
      var trailingImageCount = this.getTrailingImageCount(root, selectionState, range.endContainer, range.endOffset);
      if (trailingImageCount) {
        selectionState.trailingImageCount = trailingImageCount;
      }
      if (start !== 0) {
        var emptyBlocksIndex = this.getIndexRelativeToAdjacentEmptyBlocks(doc, root, range.startContainer, range.startOffset);
        if (emptyBlocksIndex !== -1) {
          selectionState.emptyBlocksIndex = emptyBlocksIndex;
        }...
        \end{lstlisting}
\vspace{-12pt}
\caption{An Original Code from a Project in GitHub}
\label{example_org}
\end{figure}


\begin{figure}[t]
	\centering
	\lstset{
		numbers=left,
		numberstyle= \tiny,
		keywordstyle= \color{blue!70},
		commentstyle= \color{red!50!green!50!blue!50},
		frame=shadowbox,
		rulesepcolor= \color{red!20!green!20!blue!20} ,
		xleftmargin=1.5em,xrightmargin=0em, aboveskip=1em,
		framexleftmargin=1.5em,
                numbersep= 5pt,
		language=Java,
    basicstyle=\scriptsize\ttfamily,
    numberstyle=\scriptsize\ttfamily,
    emphstyle=\bfseries,
                moredelim=**[is][\color{red}]{@}{@},
		escapeinside= {(*@}{@*)}
	}
	\begin{lstlisting}[]
function exportSelection(w, b) {
  if (!w) {
    return null;
  }
  var p = null, q = b.getSelection();
  if (q.rangeCount > 0) {
    var r = q.getRangeAt(0), d = r.cloneRange(), m;
    d.selectNodeContents(w);
    d.setEnd(r.startContainer, r.startOffset);
    m = d.toString().length;
    p = {
          start: m,
          end: m + r.toString().length
        };
    if (this.doesRangeStartWithImages(r, b)) {
       p.startsWithImage = true;
    }
    var a = this.getTrailingImageCount(w, p, r.endContainer, r.endOffset);
    if (a) {
      p.trailingImageCount = a;
    }
    if (m !== 0) {
      var y = this.getIndexRelativeToAdjacentEmptyBlocks(b, w, r.startContainer, r.startOffset);
      if (y !== -1) {
        p.emptyBlocksIndex = y;
      }...
\end{lstlisting}
\vspace{-12pt}
\caption{The Minified Code for the Code in Figure~\ref{example_org}}
\label{example_sim}
\end{figure}

Let us start with a real-world example to motivate our approach.
Figures~\ref{example_org} and~\ref{example_sim} show the original and
minified versions~of the JS function \code{exportSelection}.  The
function is aimed to export/retrieve the selection from a document.
%
In the minified code, all local variables were randomly renamed with
short and meaningless names, \eg \code{root} becomes \code{w},
\code{doc} becomes \code{b}, etc. by the minification tool,
\eg UglifyJS~\cite{uglifyJS}. This makes developers difficult to
comprehend it.


%
%The name chosen for a variable in the code should be {\em natural}
%(unsurprising) in the context~\cite{JSNaughty2017} and follow naming
%conventions~\cite{barr-codeconvention-fse14}, so that the de-minified
%code becomes easy to understand for developers.


%To achieve this goal, we conjecture that the meaningful names of
%minified variables could be observed in a large corpus of existing
%source code. This motivates us to conform our approach to a {\em
%  data-driven direction}, where we learn the names from original source
%code to recover the names for variables in the minified code.
%
%Indeed, for the minified code in Figure~\ref{example_sim}, all
%original names are found in our experimental dataset that contains
%322K JS files collected from 12K GitHub~projects.
%

\subsection{Observations}



We aim to recover the names of the variables in the minified
code. Such process is not trivial and affected by multiple factors.
Let us illustrate them via the following observations:


%1) a) variable "range"

%range.cloneRange(), range.toString()
%range.startContainer, range.startOffset

%b) variable "preSelectionRange"

%preSelectionRange.selectNodeContents(...)
%preSelectionRange.setEnd(...)

\textbf{O1}.{\em The fields and methods of a variable are kept intact
  after minification.} If the names of the fields and methods were
minified, the corresponding field accesses and method calls would not
be valid anymore. For example, \code{cloneRange()} in
\code{r.cloneRange()} at line 7 and \code{startContainer} in
\code{r.startContainer} at line 9 in Figure~\ref{example_org} are
unchanged in Figure~\ref{example_sim}. Due to that, a model can rely
on the names of those properties of a variable/object to predict the
variable's name.

\textbf{O2}. {\em The actual variable name must also be in accordance
  with the names of the accessed fields and called methods}. For
example, in the original code in Figure~\ref{example_org}, the
variable name \code{range} makes sense in \code{range.startOffset} and
\code{range.endOffset} because a range could have a start offset and
an ending offset. The rationale is that in the orignal code, for easy
code comprehension, developers tend to use the naming conventions and
meaningful names with respect to the surrounding names in the code. In
other words, the predicted name of a variable and those of its
properties (fields and methods) are in accordance. As another example,
\code{preSelectionRange} and \code{setEnd} are in accordance in
\code{preSelectionRange.setEnd} at line 9. In fact, there exist
approaches to explore the concordance between the method's name and
the names of its arguments to detect name-based bugs in a
program~\cite{deepbugs-oopsla18}.

%\textit{Each individual variable has certain properties and plays
%  particular roles in the code. Thus, the name of a variable is
%  intuitively affected by its properties and roles}. The properties
%are the method calls or field accesses to which a variable of a
%certain type can access. If a method is called or a field is accessed
%by a variable, the name of the variable should be compatible with the
%method's or the field's name. For example, in our experimental
%dataset, the number of candidates that can call method
%\texttt{getData()} (lines 6 and 14) is only 7 out of 31 variables
%names found in a function named \texttt{getClipboardContent}. Such
%number is down to a \textit{single} candidate if we additionally
%consider that it can also access the fields \texttt{getData} (line 5)
%and \texttt{types} (line 11). Thus, \texttt{r} could be named as
%\texttt{dataTransfer}, which is the same name in the original code in
%Figure~\ref{example_org}. For the variable \texttt{f} that is created
%and assigned as an element of the array \texttt{types[]} at line 13,
%there are 4 candidates for such variable that can be used as an
%argument of the method named \texttt{getData()} (line~14). The number
%of candidates for \texttt{i}, which is the returned result of the call
%to \texttt{getData()} (line 6) and also has a field with the name
%\texttt{length} (line 7), is only 7.

%2)

%range = selection.getRangeAt(0)
%preSelectionRange = range.cloneRange()

%trailingImageCount = this.getTrailingImageCount(...)

%emptyBlocksIndex = this.getIndexRelativeToAdjacentEmptyBlocks (...)

%Tien



\textbf{O3}. {\em The name of a variable can be affected by the names
  of the surrounding variables in a function.}  For example, at line
7, the choice of the name \code{range} in the recovery process could
be influenced by the choice of the variable \code{selection} as the
call to \code{getRangeAt} is made on that variable as in the statement
\code{range} = \code{selection.getRangeAt(0)}. The choice of the name
\code{preSelectionRange} could be affected by the choice of the
variable \code{range} due to the statement \code{preSelectionRange} =
\code{range.cloneRange()}.
%
Intuitively, because multiple variables are used together to achieve
the common task in the function, their names are often consistent with
one another.



%\textit{In a function, a variable might collaborate with other
%  variables to implement the function. Consequently, the recovering
%  name for a variable might be influenced by the name of
%  others}. Intuitively, since the variables are used together, their
%names are often consistent with each other to achieve the common task
%in the function.
%
%In the example, in 28 possible pairs of candidates for \texttt{i} (7
%candidates) and \texttt{f} (4 candidates), there are only 2 pairs of
%candidates that are used to name two variables in the same function
%in our dataset. One of them is the correct pair, which is
%\texttt{legacyText} and \texttt{contentType}.
%
%experimental

%3) var emptyBlocksIndex = this.getIndexRelativeToAdjacentEmptyBlocks(...);
%   if (emptyBlocksIndex !== -1) {

%Tien

\textbf{O4.} {\em The mutual impact between variable name learning and
  variable type learning.} If a model learns correctly the type of a
variable, it would help to learn better the name of the variable and
vice versa. For example, to recover the name \code{emptyBlocksIndex}
for the variable \code{y} at line 24 in Figure~\ref{example_sim}:
\code{if (y != -1)}. From that comparison, a model could learn that
\code{y} is of the type \code{int}. Knowing that it is an integer, a
model could combine that with the knowledge learned from line 23
(\code{var y = this.getIndexRelativeToAdjacentEmptyBlocks(...)}), and
predict the name for \code{y} could be something relevant to {\em
  ``Index''}. On the other hand, correct learning of a variable name
can also benefit for learning of its type. At line 18 of
Figure~\ref{example_sim}, if the name \code{trailingImageCount} is
recovered for the variable \code{a}, then its type is likely to be
\code{int} if the model could make sense of the sub-tokens
\code{image} and \code{count}.

%In contrast, if a model encounters many instances whose names are
%ended with {\em ``Index''} or {\em ``Count''}, it might be able to see
%the comparison similar to line 24 and to derive the type \code{int}
%for that variable.
%Therefore, correct learning of variable names can benefit for learning
%of variable types and vice versa.

%\textit{Within a function, e.g., \texttt{getClipboardContent}, a
%  variable name, e.g., \texttt{contentType} is affected by the
%  specific task of the function that is described by the function's
%  name~\cite{sutton-fse15}}.~This is intuitive because the names of
%variables are often~relevant~to the task that the variables are used
%in the code~to~achieve. Such task is typically described with a
%succinct function~name. In Figure~\ref{example_org}, the task of~the
%function is to get the clipboard's content, thus, it is named
%\texttt{get\-ClipboardContent}. In our~dataset, there are 31 names
%being used to specify~the variables in function
%\texttt{getClipboardContent}, \eg \texttt{data},
%\texttt{dataTransfer}, \texttt{contentType}. Meanwhile, the variable
%names \texttt{students} or \texttt{salary} have never been used in the
%function with that name.







