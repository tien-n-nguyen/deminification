\subsection{Key Ideas}
\label{key:sec}

From the above observations, we have the following key ideas.

\noindent {\bf Key Idea 1.} {\em We leverage the duality between the
  learning to predict the variable names and the learning to predict
  the variable types.} In the original code, the name of a variable
should be natural (unsurprising) with respect to the type of that
variable. We build a name prediction model and a type prediction one,
and we apply a dual-task learning mechanism between the two models.

%Moreover, if the name of a variable is recovered, its type should be
%in accordance and follow a naming convention.

%For example, at line 24, as the model encounters \code{if (y!=-1)}, it
%is expected to learn that \code{y} is of the type \code{int} and
%cannot be used in the locations where a string is expected.


\noindent {\bf Key Idea 1.} {\em A variable name is in accordance with
  the properties of the variable} including the names of the fields
and the methods in field accesses and method calls from the variable.
The names of the fields and methods are kept intact after
minification, thus, a model can rely on those names to predict the
names of the variables. For example, a model can be trained to learn
from the variables that have the field accesses to
\code{startContainer}, \code{endContainer}, \code{startOffset}, and
\code{endOffset}, and have the method calls to \code{cloneRange()},
and used to predict the variable name \code{range}.

\noindent {\bf Key Idea 2.} {\em A variable name can be influenced by
  the names or the properties of the other variables having the
  relations with that variable}. For example, a model can be trained
from the assignment \code{preSelectionRange} =
\code{range.cloneRange()} to predict the variable name on the
left-hand side of \code{X = range.cloneRange()}. In other cases, a
model can learn the names that often appear together in several
methods in the training corpus. For example, the variable names
\code{selection}, \code{selectionState}, \code{range}, and
\code{preSelectionRange}.


