Let us use Figure~\ref{rel-graph} to illustrate the benefit of {\bf
modeling the variable name generation/recovery problem as predicting
the missing features} using GCNmf~\cite{GCNmf}. A variable with a
minified name is modeled as a nodel in our graph, e.g., $r$, $q$, and
$d$.  The prior works based on machine translation (e.g.,
JSNaughty~\cite{JSNaughty2017}) aiming to translate the minified code
to the orginal code, face an issue of different naming schemes using
by different minification tools. For example, the
variable \code{range} might become $r1$, instead of $r$, by a
different minification tool or by alpha-renaming. In {\tool}, the
minified names themselves do not play a crucial role in deciding the
orignal names as in prior work. They help in recognizing the
occurrences of the same variable/object.  In our graph $G$, {\tool}
considers a node for a minified variable as a placeholder with a
missing feature that it aims to fill in.  In prediction, GCNmf will
create the vectors $v_r$, $v_q$, and $v_d$ for the nodes. During the
convolution process, those vectors are automatically updated based on
the neighboring node features and the relations. After the convolution
process, the final vectors $V_r$, $V_q$, and $V_d$ will be fed into
the GRU decoder for the variable name prediction.

%We use Figure~\ref{rel-graph} as an example to explain how our
%variable name generation model works. Because when we use different
%code minification tools to process the same source code, the minified
%variable names could be different. In this case, the minified variable
%names do not have fixed meanings in different codes with varying
%minification tools.
%So in our variable name generation model, when the generated combined
%graph comes in, we regard the node $n_r, n_q, n_d$ with the minified
%variable names such as $r, q, d$ in Figure~\ref{rel-graph} as the
%nodes with missing node features. After feeding the graph into the
%GCNmf model, it will randomly create representation vectors $v_r, v_q,
%v_d$ for nodes $n_r, n_q, n_d$ at first. During the convolution
%process, the representation vectors $v_r, v_q, v_d$ will be
%automatically updated based on the neighbor node features and the
%connection relationships. When the convolution process is finished,
%the final stage of representation vectors $v'_r, v'_q, v'_d$ for nodes
%$n_r, n_q, n_d$ are fed into the GRU decoder to predict minified
%variable names.

%At the same time, the final representation vectors for each node can
%be used in the type generation model for the type prediction.
