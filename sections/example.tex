\vspace{-4pt}
\subsubsection{Examples}
\label{sec:eval-example}

Figure~\ref{example1} shows an example of correct predictions from
{\tool}.  At line 16 (minified code), our model can derive the
type \code{bool} for \code{p} due to the \code{if} statement. At line
15, \code{p} has a relation via an assignment
with \code{check\_request\_limit()}, then our model can produce the
name \code{is\_limited} (instead of \code{limit}) for \code{p}. At
line 14, the variable \code{q} has a relation
with \code{extract\_ip\_from}, {\tool} can use the name relevant to
{\em ip} for \code{q}, e.g., \code{client\_ip}.

At line 18 with a minus operation, {\tool} can derive the type of the
variable \code{m} as \code{int}. It also has a relation
with \code{BirdNameDatabase.\-objects.\-count()}, thus, our model
can derive the name \code{count} for \code{m}.

At line 20, \code{n} is used as the index of an array type.
Moreover, at line 19, our model can derive its type of \code{int}
due to the relation with \code{randInt()}. Thus, our model
can assign the name \code{index} for \code{n}. Moreover,
due to the not-so-much-meaningful name \code{bn} at line 8,
our model did not produce the name for \code{t} at line 20.


\begin{figure}[t]
	\centering \lstset{ numbers=left, numberstyle= \tiny,
	keywordstyle= \color{blue!70},
	commentstyle= \color{red!50!green!50!blue!50},
	frame=shadowbox, rulesepcolor= \color{red!20!green!20!blue!20}
	, xleftmargin=1.5em,xrightmargin=0em, aboveskip=1em,
	framexleftmargin=1.5em, numbersep= 5pt, language=Python,
	basicstyle=\scriptsize\ttfamily,
	numberstyle=\scriptsize\ttfamily, emphstyle=\bfseries,
	moredelim=**[is][\color{red}]{@}{@}, escapeinside= {(*@}{@*)}
	} \begin{lstlisting}[]
def get(self, request):
  (*@{\color{purple}{client\_ip = self.extract\_ip\_from(request)}@*)
  (*@{\color{violet}{is\_limited = self.check\_request\_limit(client\_ip)}@*)
  (*@{\color{violet}{if is\_limited:}@*)
    return Response({}, status=rest_framework.status.HTTP_403_FORBIDDEN)
  (*@{\color{orange}{count = BirdNameDatabase.objects.count() - 1}@*)
  (*@{\color{blue}{index = randint(0, count)}@*)
  (*@{\color{blue}{bn = BirdNameDatabase.objects.all()[index]}@*)
  serialized = BirdNameSerializer(bn, many=False)
  self.save_general_statistics(client_ip, bn)
  return Response(serialized.data)
// (*@{\color{green}{Minified code}@*)
def get(self, r):
  (*@{\color{purple}{q = self.extract\_ip\_from(r)}@*)
  (*@{\color{violet}{p = self.check\_request\_limit(q)}@*)
  (*@{\color{violet}{if p:}@*)
    return Response({}, status=rest_framework.status.HTTP_403_FORBIDDEN)
  (*@{\color{orange}{m = BirdNameDatabase.objects.count() - 1}@*)
  (*@{\color{blue}{n = randint(0, m)}@*)
  (*@{\color{blue}{t = BirdNameDatabase.objects.all()[n]}@*)
  s = BirdNameSerializer(t, many=False)
  self.save_general_statistics(q, t)
  return Response(s.data)       
\end{lstlisting}
\vspace{-18pt}
\caption{A Correct Prediction Example by {\tool}}
\vspace{-4pt}
\label{example1}
\end{figure}

Figure~\ref{example2} shows another example in the project
\code{EventBus} in our dataset. At line 12, our model can recognize
the variable \code{D} used in a \code{for} loop with an array of
\code{listeners}, thus, can assign for \code{D} the name
\code{listener}. From line 8, our model can learn that \code{B} is of
the type \code{Subcribable} and at line 13 and it knows that \code{B}
is related to \code{D} (\code{listener}). Thus, from {\em Subcribable}
and \emph{listener}, it could derive the name for \code{B} as
\code{event}. At line 9, if our model can learn \code{type} is to
return the class of the variable, it can assign the name
\code{event\_class} for \code{C} due to the assignment.

\begin{figure}[t]
	\centering \lstset{ numbers=left, numberstyle= \tiny,
	keywordstyle= \color{blue!70},
	commentstyle= \color{red!50!green!50!blue!50},
	frame=shadowbox, rulesepcolor= \color{red!20!green!20!blue!20}
	, xleftmargin=1.5em,xrightmargin=0em, aboveskip=1em,
	framexleftmargin=1.5em, numbersep= 5pt, language=Python,
	basicstyle=\scriptsize\ttfamily,
	numberstyle=\scriptsize\ttfamily, emphstyle=\bfseries,
	moredelim=**[is][\color{red}]{@}{@}, escapeinside= {(*@}{@*)}
	} \begin{lstlisting}[]
def post(self, (*@{\color{orange}{event: Subscribable}@*)) -> None:
  event_class = type(event)
  if event_class not in self._listeners:
    return
  (*@{\color{violet}{for listener in self.\_listeners[event\_class]:}@*)
    (*@{\color{orange}{listener(event)}@*)

def post(self, (*@{\color{orange}{B:Subscribable}@*))->None:
  C=type(B)
  if C not in self._listeners:
    return
  (*@{\color{violet}{for D in self.\_listeners[C]:}@*)
    (*@{\color{orange}{D(B)}@*)
\end{lstlisting}
\vspace{-16pt}
\caption{Another Correct Prediction Example in Dataset}
\label{example2}
\end{figure}

