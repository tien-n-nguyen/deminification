\begin{figure}[t]
	\centering
	\lstset{
		numbers=left,
		numberstyle= \tiny,
		keywordstyle= \color{blue!70},
		commentstyle= \color{red!50!green!50!blue!50},
		frame=shadowbox,
		rulesepcolor= \color{red!20!green!20!blue!20} ,
		xleftmargin=1.5em,xrightmargin=0em, aboveskip=1em,
		framexleftmargin=1.5em,
                numbersep= 5pt,
		language=Python,
    basicstyle=\scriptsize\ttfamily,
    numberstyle=\scriptsize\ttfamily,
    emphstyle=\bfseries,
                moredelim=**[is][\color{red}]{@}{@},
		escapeinside= {(*@}{@*)}
	}
	\begin{lstlisting}[]
function exportSelection(self, root, doc) {
  if not root: 
      return null
  selection = doc.getSelection()
  if selection.rangeCount > 0:
      range_ = selection.getRangeAt(0)
      preSelectionRange = range_.cloneRange()
      preSelectionRange.selectNodeContents(root)
      preSelectionRange.setEnd(range_.startContainer, range_.startOffset)
      start = len(str(preSelectionRange))
      selectionState = {
          "start": start,
          "end": start + len(str(range_))
      }
      if self.doesRangeStartWithImages(range_, doc):
          selectionState.startsWithImage = true
      trailingImageCount = self.getTrailingImageCount(root, selectionState, range_.endContainer, range_.endOffset)
      if trailingImageCount:
        selectionState.trailingImageCount = trailingImageCount;
      if start != 0:
        (*@{\color{violet}{emptyBlocksIndex = self.getIndexRelativeToAdjacentEmptyBlocks}@*)(doc, root, range_.startContainer, range_.startOffset)
        (*@{\color{violet}{if emptyBlocksIndex != -1:}@*) 
          selectionState.emptyBlocksIndex = emptyBlocksIndex
        ...
        \end{lstlisting}
\vspace{-12pt}
\caption{An Original Code from a Project in GitHub}
\label{example_org}
\end{figure}
