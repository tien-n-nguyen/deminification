\subsection{Experimental Methodology}
\label{empirical-method}

\subsubsection{RQ1. Comparative Study on Variable Name Prediction (Python)\\}
{\em Baselines.} We compared {\tool} against the state-of-the-art
variable name recovery approaches for minified code including
JSNeat~\cite{icse19}, JSNice~\cite{JSNice2015}, and
JSNaughty~\cite{JSNaughty2017}.

\textit{Procedure.} We took all the methods in that Python dataset
and used Pyminifier to produce the minified code with variable name
mininification. We randomly split all the methods into 80\%, 10\%,
10\% in which 80\% of the methods as the training dataset, 10\% of the
methods as the tuning dataset, and 10\% of the methods as the testing
dataset for all baselines and {\tool}.

%We took all the methods in the ManyTypes4Py dataset and ran the pyminifier \cite{pyminifier} tool that is used for variable name minification to minify all the variables in each method. Then we use 80\% of the methods as the training dataset, 10\% of the methods as the tuning dataset, and 10\% of the methods as the testing dataset for all baselines and {\tool}.

{\em Parameter Tuning.} We tuned {\tool} with autoML~\cite{NNI} for
the following key hyper-parameters to have the best performance: (1)
Epoch size (100, 150, 200); (2) Batch size (64, 128, 256); (3)
Learning rate (0.001, 0.005, 0.010); (4) Vector length of feature
embeddings and its output (32, 64, 128); (5) Number of GCN layers (4,
8, 16).

{\color{blue}{
    \subsubsection{RQ2. Comparative Study on Variable Name Prediction (JS)\\}

We build the JS representation graph and use the same model.
    
{\em Baselines.} We compared {\tool} against the state-of-the-art
approaches JSNeat~\cite{icse19}, JSNice~\cite{JSNice2015}, and
JSNaughty~\cite{JSNaughty2017}.

%variable name recovery approaches for minified code including
%JSNeat~\cite{icse19}, JSNice~\cite{JSNice2015}, and
%JSNaughty~\cite{JSNaughty2017}.

\textit{Procedure and Tuning.} We used UglifyJS for code minification
in the JS dataset. We used the same splitting and tuning as in RQ1.


}}

%with variable name mininification. We randomly split all the methods
%into 80\%, 10\%, 10\% in which 80\% of the methods as the training
%dataset, 10\% of the methods as the tuning dataset, and 10\% of the
%methods as the testing dataset for all baselines and {\tool}.

%{\em Parameter Tuning.} We tuned {\tool} via autoML~\cite{NNI} as in RQ1.}}
%on the same set of parameters as in RQ1.}}
%for the following key hyper-parameters to have the best performance:
%(1) Epoch size (100, 150, 200); (2) Batch size (64, 128, 256); (3)
%Learning rate (0.001, 0.005, 0.010); (4) Vector length of feature
%embeddings and its output (32, 64, 128); (5) Number of GCN layers (4,
%8, 16).

\subsubsection{RQ3. Comparative Study on Variable Type Prediction\\}

{\em Baselines.} We compared {\tool} against the state-of-the-art
  variable type prediction approaches: Ivanov {\em et
  al.}~\cite{ivanov21predicting}, Typilus~\cite{typilus-pldi20},
  TypeWriter\cite{typewriter-fse20}, Type4Py~\cite{Type4Py-icse22},
  and HiTyper~\cite{HiTyper-icse22}.

%HiTyper~\cite{type-graph-icse22}, Type4PY~\cite{mir2021type4py},
%Typilus~\cite{allamanis2020typilus}, Ivanov et
%al.\cite{ivanov21predicting}, and TypeWriter~\cite{typewriter-fse20}.

We did not compare our tool with Xu {\em et al.}~\cite{xu-fse16},
DeepTyper~\cite{DeepTyper-fse18}, NL2Type~\cite{nl2type-icse19},
LAMBDANET~\cite{LambdaNet-iclr20}, OptTyper~\cite{optyper20}, and
TypeBERT~\cite{typeBERT-fse21}, because in its paper, Type4Py has been
shown to perform better than those tools.
%, and we are comparing against Type4Py.
We did not compare on JS code because JS variables must not always
have type declarations.


\textit{Procedure.}  With all variable type information summarized in
the ManyTypes4Py dataset, we directly used the type information for
the variables as the ground truth for all the approaches. We used the
same data splitting for training, tuning, and testing as in RQ1. We
ran the baselines on the original source code with the parameters in
their documentation. We fed to {\tool} the minified code but with the
original names and no types (i.e., the original code), and obtained
the types from the variable type generation (VTG) model for all the
variables/expressions in the code.

%Tien paid attention to this
%We fed to DeMinify the minified code but with ***all the names***

{\em Parameter Tuning.} Tuning was done via
autoML~\cite{NNI} as in RQ1.
%for the same hyper-parameters to have the best performance as in RQ1. 

%Because {\tool} is a dual-learning approach, to make {\tool} work
%correctly, we took the same data splitting for training, tuning, and
%testing dataset as in RQ1 for all baselines and {\tool} in this RQ.
%Because all baselines in this RQ are designed for python, we directly
%used the original implementation for each baseline and kept all
%parameters the same as in their original research studies. And we
%tuned {\tool} with autoML~\cite{NNI} for the same key hyper-parameters
%to have the best performance as in RQ1.

\subsubsection{RQ4. Ablation Study}

We built several variants of {\tool} to evaluate different factors in
its components by measuring their accuracies and making comparisons.
First, we evaluate our hypothesis that mutual impact between name
learning and type learning via dual-task learning. We built two
variants VNG and VTG separately without connecting them via dual-task
learning. Second, the graph representation for source code is
important in the name and type prediction. Thus, we kept the same
architecture for {\tool} and fed to it different graph representations
for source code. This allows us to evaluate the impact of our chosen
graph representations for this problem. Third, the key technical
solution in {\tool} is the formulation in which predicting names and
types for variables is considered as predicting the missing features
of connected program elements. We currently used the EE-GCN model to
capture the relations and dependencies among program elements. We
replaced it with the Label-GCN~\cite{label-gcn} and made a comparison.

%We built the variants of {\tool} by changing its important components to check the influence of each component on both the variable name and type prediction results. These variants of {\tool} include: 1. using label-GCN (LGCN) as the variable type prediction model and using GCN with missing features (GCNmf) as the variable name prediction model on relation graph (RG); 2. using edge-enhanced GCN (EEGCN) as the variable type prediction model and using GCNmf as the variable name prediction model on code property graph (CPG); 3. using EEGCN as the variable type prediction model and using GCNmf as the variable name prediction model on RG; 4. using EEGCN as the variable type prediction model and using GCNmf as the variable name prediction model on type dependency graph (TDG); 5. using EEGCN as the variable type prediction model and using GCNmf as the variable name prediction model on the combined graph (CG).

\subsubsection*{Evaluation Metrics}

For the variable name prediction, we follow a prior work~\cite{icse19}
to calculate the prediction accuracy on local~variables and all
variables. We compared the resulting names from a tool against the
original names. A tool is considered to correctly
recover the name of a variable $v$ if the recovered name $vn$ is
matched exactly with its original name. For $v$, if matching, we count
it as a hit, otherwise, it is a miss. Accuracy is measured by the
ratio between the total number of hits over the total number of
cases. Top-$k$ accuracy is measured similarly, however, a hit is
achieved when the correct name is in the top-$k$ candidate list from a
tool.

For the variable type prediction, we use two metrics,
\code{Exact-Match} and \code{Parametric-Match} with the top-$k$
accuracy as in HiTyper~\cite{HiTyper-icse22}. \code{Exact-Match}
occurs when the resulting type matches exactly with the
human-annotations. \code{Parametric-Match} occurs if the result
matches with the correct type but those of the parameters might
not. For example, \code{Dict[int,str]} is 
parametric-matched with \code{Dict[int,int]}.

%These two metrics compute the ratio of results that: 1) Exact Match:
%completely matches human annotations. 2) Match to Parametric: satisfy
%exact match when ignoring all the type parameters. (e.g., Dict[int,
%str] and Dict[int, int] are considered as matched under Match to
%Parametric metric.)

