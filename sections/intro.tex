\section{Introduction}
\label{intro:sec}

Understandability is an important software quality. Software
developers have to spend a significant portion of their efforts in
reading and comprehending source code. Beside the documentation,
meaningful names for variables and types are crucial for developers in
quickly grasping the essence of the code. Software organizations have
much emphasized on naming conventions and coding standards to ensure
meaningful variable names in source
code~\cite{barr-codeconvention-fse14}.

%An important aspect of program understanding is the names of the
%identifiers~used in the source code~\cite{sutton-fse15}. Meaningful
%identifiers help developers tremendously in quickly grasping the
%essence of the~code. Thus, naming conventions are strongly emphasized
%on prescribing how to choose meaningful variable names in coding
%standards~\cite{barr-codeconvention-fse14}.  These principles also
%apply to Web development.

%In modern Web development, program understanding plays an equally
%important role. 

In modern software development, some technologies require the exposure
of source code, e.g., in Web programming languages, the JavaScript
(JS) client code need to executed in the client side.

Web technologies and programming languages require the exposure of
source code to Web browsers in the client side to be executed
there. To avoid such exposure, the source code such as JavaScript (JS)
files are often obfuscated in which the variable names are minified,
\ie the variable names are replaced with short, opaque, and
meaningless names. The intention has two folds. First, it makes the JS
files smaller and thus is quickly loaded for better performance.
Second, minification diminishes code readability to hide business
logics from the readers, while maintaining the program semantics.
%
%\textbf{would be better to focus on the reason of making files small. In Web,JS code is transfer over the Internet from server to client to be executed there. Due to the limit in the bandwidth or the memory capacity of the 
%(mobile) device, JS code is normally minified to reduce its size. When being minified, the names are replaced ...}
%

Due to those reasons, there is a natural need to automatically recover
the minified code with meaningful variable names. When the original
code is not available, with such recovery, the minified JS code will
be made accessible for code~compre-\\hension as well as other maintenance
activities such as~code review, reuse, analysis, and
enhancement. Recognizing that need, researchers have been introducing
the automatically recovering tools for variable names in JS code.
%
JSNice~\cite{JSNice2015} is an automatic variable name recovery
approach that represents the program properties and relations among
program entities in a JS code as dependence graphs.
%
It leverages advanced machine learning (ML) to recover missing
variable names.
%Final
%It leverages an advanced machine learning (ML) model named Conditional
%Random Fields (CRF) to recover the variable names, which are modeled
%as the missing information in a CRF.
%
%Final
%With the same vein of
Using also ML, JSNaughty~\cite{JSNaughty2017}
formulates the variable name recovery problem for JS code as a
%phrase-based
statistical machine translation from minified code to the
recovered code. Despite of their successes, both approaches still
suffer low accuracy and scalability issues with the use of
computationally expensive ML algorithms.
