\section{Related Work}
\label{related_section}

%The automated approaches for variable name recovery in the minified
%source code can be broadly classified

\noindent {\bf Variable name recovery approaches} in minified code
follow the three categories: {\em information retrieval} (IR),
statistical learning, and {\em machine learning} (ML).
JSNeat~\cite{icse19} follows an IR approach to search for the names in
large code corpus, but is not effective for un-seen names.
%with the idea that the names of a minified variable could be found in
%an ultra-large-scaled code corpus.  The key issue is that JSNeat is
%not effective if a name has never been seen before in the training
%corpus.
JSNice~\cite{JSNice2015}
%uses the graph representation of variables
%and surrounding program entities via program dependencies. It
infers the variable names via structured prediction with
conditional random fields (CRFs)~\cite{JSNice2015}. We showed that
deriving missing features via a neural network yields higher accuracy
than predicting program properties with statistical learning.
{\tool} is computationally heavier than those approaches.

%IR-based and statistical
%learning-based solutions.

JSNaughty~\cite{JSNaughty2017} uses a statistical machine
translation from the minified code to recovered code. First, it
faces the issue of relying on minified names as explained.
%different minification tools using different naming schemes.
%because it considers the minified
%names as the target sentences in their machine translation.
Second, it uses a phrase-based translation model, which
enforces a strict order between the recovered variable names in a
function.
%This is too strict since a name of a variable might not need
%to occur before another name of another variable.  In contrast,
The other methods for code deobfuscation mainly leverage
static/dynamic
analyses~\cite{Christodorescu:2003:SAE:1251353.1251365,Moser:2007:EME:1263552.1264210,Udupa05deobfuscation:reverse}.



\noindent {\bf ML-based Type Inference Approaches.}
HiTyper~\cite{HiTyper-icse22} is a hybrid approach between static
inference and deep learning. It uses TDG to encode type inference
rules to conduct type rejection to inspect the output predictions. It
iteratively conducts static inference and DL-based prediction until
the TDG is fully inferred.
%In comparison, {\tool} focuses on variable
%name recovery for minified code, and leverages TDG with EE-GCN to
%infer the types, improving the name generation in a dual-task learning
%fashion.
As shown, {\tool} also improves over HiTyper due to the propagation of
VNG to VTG.
%
Type4Py~\cite{Type4Py-icse22} is a deep similarity learning-based
hierarchical neural network model. It learns to discriminate between
similar and dissimilar types in a high-dimensional space.
%Likely types are then inferred through the nearest neighbor search.
We formulate the problem as missing feature recovery,
avoiding finding similar types via embeddings.
%rather than finding similar types in high-dimensional space.
Similarly, Typilus~\cite{typilus-pldi20} proposes TypeSpace
containing the embeddings with type properties of a
symbol. TypeWriter~\cite{typewriter-fse20} learns to infer the return
and argument types for functions from partially annotated code bases
by combining the natural language properties of code with programming
languages. {\tool} does not use natural-language
properties. Ivanov {\em et al.}~\cite{ivanov21predicting} demonstrate
that graph-based embeddings could improve the quality of type
prediction. We use GCNmf to predict the missing features.

%%
%Third, JSNaughty does not consider the task context of the
%variables. Finally, our training/testing time is much faster.

%{\tool} is closely related JSNice~\cite{JSNice2015} and
%JSNaughty~\cite{JSNaughty2017}.  JSNice~\cite{JSNice2015} uses the
%graph representation of variables and surrounding program entities via
%program dependencies. It infers the variable names as a problem of
%structured prediction with conditional random fields
%(CRFs)~\cite{JSNice2015}. In comparison, first, while JSNice uses ML,
%{\tool} is IR-based in which it searches for a list candidate names in
%a large code corpus. Second, {\tool} considers not only the impacts of
%surrounding program entities in SVC, but also task and
%multiple-variable contexts.  Third, with CRF, JSNice is effective when
%variables have more dependencies, and less effective~with the
%functions having one variable. Finally, {\tool} is much faster and the
%results are more accurate as shown in
%ection~\ref{empirical_result_section}.

%Statistical NLP approaches have been used in SE.
%Allamanis {\em et al.}~\cite{sutton-fse15} propose a neural
%probabilistic language model to suggest method/class names.
%Tien
%The code tokens with statistical co-occurrences are projected into a
%continuous space together with the text tokens from the names.
%
Statistical NLP approaches have been used for name and code style
suggestions~\cite{sutton-fse15,barr-codeconvention-fse14}.
%by providing better names.
%Their specialized model handles well the method/class naming problem
%and is not suitable to capture the relations among (much more) texts
%and code elements in software documentation.
%
%Allamanis {\em et al.}~\cite{bimodal15} introduce a jointly
%probabilistic model short natural language utterances and source code
%snippets.
%^Researchers have applied {\em statistical NLP methods} including word
%embeddings to software artifacts.
% Tien
%PAM~\cite{sutton-16} is a parameter-free, probabilistic algorithm to
%mine API patterns.
%It uses a probabilistic formulation of frequent sequence
%mining on API sequences.
%Allamanis {\em et al.}~\cite{sutton-fse15}
%suggest methods/classes' names using embeddings.
%Code elements' names are broken down into words.
%The elements with statistical cooccurrences are projected into a
%continuous space with the words from the names.
%The model learns which names are semantically similar by assigning
%them to locations such that names with similar embeddings tend to be
%used in similar contexts~\cite{sutton-fse15}.
%In comparison, we use Word2Vec~and learn the transformation
%between two spaces. Their model works in the same
%space. 
%Moreover, {\tool} works on the abstraction level of API elements,
%rather than names of tokens.
%Maddison and Tarlow~\cite{tarlow14} present a generative model for
%source code that is based on AST
%structures.
%Tien
%TBCNN~\cite{tbcnn14} also uses trees for suggest next code
%tokens.
%
Other applications 
include code suggestion~\cite{hindle-icse12,tbcnn14}, code
convention~\cite{barr-codeconvention-fse14}, name
suggestion~\cite{sutton-fse15}, API suggestions~\cite{raychev-pldi14},
code mining~\cite{sutton-msr13}, type resolution~\cite{icse18},
pattern mining~\cite{sutton-16}, code generation
%Statistical NLP was used to generate code from text,
\eg SWIM~\cite{Raghothaman-ICSE16},
DeepAPI~\cite{gu-fse16}, Anycode~\cite{anycode-oopsla15}.

%first uses IBM Model with word
%translation to produce code elements. It then uses syntactic rules on
%those elements to build code sequences close to the
%query. DeepAPI~\cite{gu-fse16} uses RNN to generate API sequences for
%a given text by using deep learning to relate APIs. Desai {\em et
%  al.}~\cite{Desai2016ICSE} synthesize domain-specific languages from
%English. Anycode~\cite{anycode-oopsla15} uses a probabilistic CFG
%with trees for Java constructs and API calls to synthesize small Java
%expressions. T2API~\cite{icsme18} uses graph-based API synthesis
%algorithm that generates a graph representing an API usage from a
%large corpus.
