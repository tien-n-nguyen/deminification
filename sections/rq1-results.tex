\subsection{{\bf RQ1. Comparative Study on Variable Name Prediction.}}
\label{empirical-rq1}

\begin{table}[t]
	\caption{RQ1. Comparative Study on Variable Name Prediction.}
	\begin{center}
		\small
		\renewcommand{\arraystretch}{1} 
		\begin{tabular}{p{2cm}<{\centering}|p{2cm}<{\centering}|p{2cm}<{\centering}}
			\hline
		                & Local Variables & All Variables\\
			\hline
			JSNeat      & 0.64            & 0.66    \\
			JSNice      &                 &         \\
			JSNaughty   &                 &         \\
			\hline
			{\tool}     & 0.67            & 0.76    \\
			\hline
		\end{tabular}
		\label{RQ1-result}
	\end{center}
\end{table}

1. On both local variables and all variables, {\tool} performs XXX\% better than the baselines. 
2. JSNeat uses information retrieval techniques to find the possible variable names based on the training data. {\tool} can perform better because sometimes it is hard to find similar variable names with similar variable relationships, and sometimes the same variable names may have different variable relationships. JSNeat cannot perform well in these cases, but {\tool} can still work as designed.
3. JSNice uses conditional random field models to predict variable names based on dependency graphs. {\tool} can outperform {\tool} by XX\% because {\tool} uses combine graph from relation graph and TDG. The combined graphs contain more detailed information between variables compared with dependency graphs. 
4. JSNaughty regards the name prediction problem as the translation problem. {\tool} can outperform JSNaughty XX\% because different minification tools can provide different minified variable names for the same source code, but {\tool} regards the minified variable names as missing features in GCNmf. In this case, the minified variable names will not influence the performance of {\tool}, but it will influence JSNaughty performance. 